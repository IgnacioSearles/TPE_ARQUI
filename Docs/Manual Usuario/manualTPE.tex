\documentclass{article}
\title{\textbf{Manual de Usuario - Trabajo Práctico Especial (TPE)}}
\author{ \large Instituto Tecnológico de Buenos Aires - Arquitectura de las computadoras (72.08) \\ [1ex]
\large Grupo 21}
\date{4 de junio de 2024}

\usepackage{multicol}
\usepackage{graphicx, wrapfig}
\graphicspath{ {imagenes/} }

\usepackage{float}
\usepackage{amsmath}
\usepackage{amsfonts}

\usepackage{hyperref}

\usepackage{caption, threeparttable}
\usepackage{hyperref}

\usepackage[margin=1.3in]{geometry}

\usepackage{listings}

\renewcommand{\figurename}{Figura}
\renewcommand{\tablename}{Tabla}
\renewcommand*\abstractname{Resumen}

\begin{document}
\maketitle

\section *{Ejecución}

Una vez compilado el proyecto y generada la imagen, se puede ejecutar el mismo mediante el siguiente comando.

\begin{lstlisting}[language=bash]
$> ./run.sh
\end{lstlisting}

Notar que el proyecto utiliza flags de qemu para poder emitir sonido. Los flags necesarios para esto cambiaron entre versiones de qemu.

\section *{Shell}

El shell cuenta con varios módulos que se pueden ejecutar. Al iniciarse el shell por primera vez se ejecuta el módulo help. El módulo help lista todos los módulos disponibles, junto con una descripción de su funciónalidad.

Lista de comandos:
\begin{itemize}
\item help: muestra los módulos disponibles.
\item clear: limpia el buffer de texto de la pantalla.
\item sysinfo: muestra información del sistema.
\item beep: emite un sonido.
\end{itemize}

\section *{Desarrollador}

El shell permite facilmente agregar nuevos módulos. Para crear un modulo es necesario agregar una entrada al array de módulos que contiene structs de ModuleDescriptor.

\begin{lstlisting}[language=C]
typedef struct {
    char* module_name;
    char* module_description;
    void (*module)();
} ModuleDescriptor;
\end{lstlisting}

El module\_name es el que será utilizado como nombre del comando en la shell. El module\_description es usado por el comando help. Y el tercer elemento del struct es un puntero a la función del módulo.

Ya existen implementadas las siguientes funciones en la librería estandar: putchar, puts, printf, strlen, strcmp, strcpy, atoi, itoa, getchar, scanf. Y existen adaptadores para los syscalls.

Los syscalls disponibles se listan en la tabla \ref{table:syscalls}. Están modelados a partir de la API de Linux 64bit.

\begin{center}
\hspace*{-0.85in}
\begin{tabular}{|c|c|c|c|c|c|c|}
\hline
\textbf{rax} & \textbf{syscall} & \textbf{rdi} & \textbf{rsi} & \textbf{rdx} & \textbf{r10} & \textbf{r8} \\ \hline
0 & sys\_read &  const char* buff & uint64\_t len & & & \\ \hline
1 & sys\_write & uint64\_t fd & const char* buff & uint64\_t len & &  \\ \hline
2 & sys\_put\_text & const char* str & uint32\_t len & uint32\_t hexColor & uint32\_t posX & uint32\_t posY \\ \hline
3 & sys\_set\_font\_size & uint32\_t font\_size & & & &  \\ \hline
4 & sys\_draw\_square & uint32\_t hexColor & uint32\_t posX & uint32\_t posY & uint32\_t size &  \\ \hline
5 & sys\_get\_screen\_width & & & & & \\ \hline
6 & sys\_get\_screen\_height & & & & &   \\ \hline
7 & sys\_get\_time & & & & &   \\ \hline
8 & sys\_get\_key\_pressed & & & & &   \\ \hline
9 & sys\_get\_character\_pressed & & & & &   \\ \hline
10 & sys\_clear\_text\_buffer & & & & &   \\ \hline
11 & sys\_get\_cpu\_vendor & char* buff & & & &   \\ \hline
12 & sys\_beep & uint64\_t freq & uint64\_t secs & & &  \\ \hline
12 & sys\_delay & uint64\_t millis & & & &  \\ \hline
\end{tabular}
\captionof{table}{syscalls implementadas para el llamado de funciones de kernel desde el userspace}
\label{table:syscalls}
\end{center}

\end{document}
